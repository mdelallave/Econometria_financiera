

' Creamos un fichero cuyo sample se adecue a nuestros datos
wfcreate(wf=entrega) m 1/07/1926 2015

' Importamos las series de Excel
read(t=xls,b2) hitec.xls hitec
read(t=xls,c2) hitec.xls rmrf
read(t=xls,d2) hitec.xls smb
read(t=xls,e2) hitec.xls hml
read(t=xls,f2) hitec.xls rf

' Ej.1: 
output(x) ej1 ' Fichero de salida para los datos de interés.
equation eq1.ls(p) hitec-rf c rmrf smb hml ' Ajuste de regresion por MCO 

' Ej. 2:
output(x) ej2
eq1.auto(2)(p)   ' Contraste de autocorrelacion del modelo con 2 lags.
eq1.white(p) ' Test de heterocedasticidad.
eq1.hist(p) ' Realizamos el histograma para comprobar la normalidad.

' Ej. 3:
output(x) ej3
eq1.wald(p) c(3) = c(4) = 0' Test de Wald

' Ej. 4:
output(x) ej4
equation eq2.ls(p) hitec-rf c rmrf smb hml smb^2
eq2.wald(p) c(5) = 0

' Ej. 5:
output(x) ej5
series d1946=@date>@dateval("1945m12") ' Dummy variable que 
' elimina los datos previos a 1946
equation eq3.ls(p) hitec-rf c rmrf smb hml d1946*rmrf
eq3.wald(p) c(5) = c(2)

' Ej. 6:
output(x) ej6
equation eq1_white.ls(cov=white, p) hitec-rf c rmrf smb hml 
' Ajuste de regresion por MCO solcitando que la matriz de var-cov
' cumpla el test de White (sea robusta).
equation eq1_hc.ls(cov=hc, p) hitec-rf c rmrf smb hml 
' Igual pero siguiendo el criterio de Heterokedasticity autocorrelation consistent
' (más usado). 
